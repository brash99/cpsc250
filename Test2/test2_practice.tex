\documentclass[11pt]{article}
\usepackage{amsmath}
\usepackage{amssymb}
\usepackage{listings}
\usepackage{fancyhdr}
\usepackage{geometry}
\geometry{margin=1in}
\usepackage{enumitem}
\usepackage{multicol}

\pagestyle{fancy}
\fancyhf{}
\rhead{CPSC 250 – Test 2}
\lhead{Programming for Data Manipulation}
\cfoot{\thepage}

\title{\vspace{-1cm}CPSC 250 – Programming for Data Manipulation – Test 2}
\date{}
\begin{document}
\maketitle

\noindent\textbf{Time:} 75 minutes \\
\textbf{Instructions:} Answer all questions on this test paper. No electronic devices or notes are permitted. Be sure to write clearly.

\vspace{0.5cm}
\section*{Part A: Multiple Choice (10 points, 2 points each)}
\noindent Circle the one correct answer.

\begin{enumerate}[label=\arabic*.]
    \item What happens when you pass a mutable object like a list to a Python function? \\
    \begin{tabular}{ll}
        A. & A new copy is created \\
        B. & It behaves like pass by value \\
        C. & The function can modify the original list \\
        D. & The function cannot access the list
    \end{tabular}

    \item Which of the following best describes what \verb|__str__()| is used for? \\
    \begin{tabular}{ll}
        A. & Converts all strings in a class to uppercase \\
        B. & Returns a string representation of an object \\
        C. & Initializes an object \\
        D. & Deletes a string from memory
    \end{tabular}

    \item In Python, when does garbage collection typically occur? \\
    \begin{tabular}{ll}
        A. & When the CPU is idle \\
        B. & When a variable is reassigned \\
        C. & When an object's reference count reaches zero \\
        D. & At the end of a function call
    \end{tabular}

    \item What is the purpose of a setter method? \\
    \begin{tabular}{ll}
        A. & To display object attributes \\
        B. & To initialize instance variables \\
        C. & To allow controlled access to modify private variables \\
        D. & To overload arithmetic operators
    \end{tabular}

    \item Which statement about operator overloading is true? \\
    \begin{tabular}{ll}
        A. & You cannot redefine \verb|+| in Python \\
        B. & Operator overloading is only for strings \\
        C. & You use special methods like \verb|__add__()| to define operator behavior \\
        D. & Operator overloading requires C++
    \end{tabular}
\end{enumerate}

\newpage

\section*{Part B: Find the Error (15 points, 3 points each)}
Each of the following code snippets has one or more errors. Identify and explain them.

\begin{enumerate}[label=\arabic*.]
    \item \begin{lstlisting}[language=Python]
def change_value(x):
    x = x + 1

y = 10
change_value(y)
print(y)  # Expected output: 11
    \end{lstlisting}

    \item \begin{lstlisting}[language=Python]
class Circle:
    def __init__(radius):
        self.radius = radius
    \end{lstlisting}

    \item \begin{lstlisting}[language=Python]
def append_item(mylist=[]):
    mylist.append(1)
    return mylist
    \end{lstlisting}

    \item \begin{lstlisting}[language=Python]
class Dog:
    def __init__(self, name):
        self.name = name
    
    def __str__(self):
        return self.name

buddy = Dog("Buddy")
print(buddy.__str__)
    \end{lstlisting}

    \item \begin{lstlisting}[language=Python]
class Point:
    def __init__(self, x, y):
        self.x = x
        self.y = y

    def __add__(other):
        return Point(self.x + other.x, self.y + other.y)
    \end{lstlisting}
\end{enumerate}

\newpage

\section*{Part C: Code Writing (30 points)}

\begin{enumerate}[label=\arabic*.]
    \item (10 points) Write a function \verb|swap_values(a, b)| that attempts to swap two variables. Then write a short explanation as to \textbf{why or why not} the values change outside the function.
    
    \vspace{10cm}

    \item (10 points) Write a class called \verb|Rectangle| that has:
    \begin{itemize}
        \item Two instance variables: \verb|width| and \verb|height|
        \item A constructor
        \item Getters and setters for both variables
        \item A method \verb|area()| that returns the area
        \item A \verb|__str__()| method that returns a string like \verb|"Rectangle(width=3, height=4)"|
    \end{itemize}
    
    \vspace{6cm}
    
    \newpage

    \item (10 points) Extend your \verb|Rectangle| class to allow adding two rectangles using \verb|+| so that \verb|r3 = r1 + r2| creates a new rectangle with combined width and height.

    \vspace{10cm}
\end{enumerate}

\section*{Part D: Code Commentary (20 points)}  
The following program creates a class representing a simple bank account. \textbf{Write comments next to each line} explaining what it does.

\begin{lstlisting}[language=Python]
class BankAccount:
    def __init__(self, owner, balance=0):
        self.owner = owner
        self.balance = balance

    def deposit(self, amount):
        self.balance += amount

    def withdraw(self, amount):
        if amount <= self.balance:
            self.balance -= amount
        else:
            print("Insufficient funds")

    def __str__(self):
        return f"{self.owner}'s account balance: ${self.balance}"
\end{lstlisting}

\end{document}
